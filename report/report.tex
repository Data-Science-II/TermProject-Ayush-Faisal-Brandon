\documentclass{article}

\usepackage{geometry}
\geometry{letterpaper, total={7in, 10in} }

\usepackage{hyperref}
\usepackage{booktabs}
\usepackage{graphicx}
\usepackage{verbatim}
\usepackage{appendix}

\title{Term Project CSCI 4360, Housing Market Forecasting}
\author{Ayush Kumar, Faisal Hossain, Brandon Amirouche}
\date{April 27, 2021}

\begin{document}
	\maketitle
	\tableofcontents
	\newpage
	
	\section{Problem Statement: Forecasting Housing Market}
	
	The goal of our project was to determine if future housing market pricing can be explained by 
	previous housing market data. We also wanted analyze the usage of exogenous variables in our time-series 
	modeling, and whether it makes a difference. This task is distinct from housing market prediction, which 
	relies on the features of the market, and the house. Using time-series data presented our group a new 
	and novel challenge. 
	
	\section{The Datasets}
	
	We identified two datasets for potential use in our projects. 
	
	\begin{enumerate}
		\item  \href{https://www.quandl.com/databases/ZILLOW/data}{Quandl Data from Zillow}
		\item \href{https://www.consumerfinance.gov/data-research/hmda/historic-data/?geo=nationwide&records=all-records&field_descriptions=labels}{USA Home Mortgage Disclosure Act Data}
	\end{enumerate}

	Each dataset has very different forms, and presented different challenges. The Housing Mortgage Disclosure Act was passed 
	by congress in 1975, and requires multiple federal agencies to keep track of all mortgage loans filed for and 
	denied. The datasets that we are using from the HMDA contain all mortgages filed for in America for a given year. 
	This data is very large, and we decided to use only a subset of the the available data from 2014-2017. These 
	4 datasets combined were around 40GB of data, and so we required a different approach for data analysis. We 
	got around the large nature of the data using an SQLite3 database and querying it for EDA using the corresponding 
	sqlite3 python library. 
	
	The Zillow data comes from Quandl, a web api for many types of time-series data. This data has over 70,000 regions
	and corresponds to years of aggregated monthly home sales. For the purposes of this project we downloaded 43 different 
	regional data, mostly from the state of Washington. These different regions will be used for testing our models, and we will try 
	to compare them to one another. A key drawback of this approach is that some regions have data that goes further back 
	than other regions, as long as the future housing values aren't affected by values from further in the past than 10+ years
	the results should be directly comparable. 
	
	We will decide on which datasets to use for our modeling after taking into account the preprocessing that we will need, and 
	doing some exploratory data analysis. 

	\section{Preprocessing Techniques}
	\section{Exploratory Data Analysis}
	\section{Modeling}
	\subsection{Auto-Regressive Model}
	\subsection{Seasonal Auto-Regressive Integrated Moving Average}
	\subsection{Gated Recurrent Unit RNN}
	\subsection{Long Short-Term Memory}
	\section{Results \& Conclusion}
	\section{Recommendations of Study}
	
\end{document}